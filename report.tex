\documentclass{article}
\usepackage[a4paper, margin=2cm]{geometry}
\usepackage{graphicx}
\usepackage{algorithm}
\usepackage{algpseudocode}
\usepackage{hyperref}
\usepackage{comment}
\usepackage{amsmath}
\usepackage{enumitem}

\newtheorem{theorem}{Theorem}[section]   % numbered per section
\newtheorem{lemma}[theorem]{Lemma}       % shares numbering with theorem
\newtheorem{corollary}[theorem]{Corollary}
\newtheorem{definition}[theorem]{Definition}
\usepackage{titling}

% Title formatting
\pretitle{\centering\LARGE\bfseries\vspace{0em}}
\posttitle{\par\vspace{1em}}

\preauthor{\centering\large}
\postauthor{\par\vspace{1em}}

\predate{\centering\small}
\postdate{\par}


\setlength{\parindent}{0pt} % no indend automatically everywhere

\title{Project Report\\Binary Heap and Dijkstra's Algorithm}
\author{
	\centering
	Josefine Lindmar
	Sonja Joost
}
\date{\today} % empty date



\begin{document}

\maketitle

\section{Introduction}
Explain high level idea, name all challenges addresed in the upcoming sections

\section{Binary Heap}
Array vs Tree

\section{Dijkstra's algorithm}
implementation, termination proof, representation of the data, the proof, which helper lemmas are still needed 


full / in detail paper proof, always connecting to the progress in Lean

\section{Reflection on LLM Usage}

\begin{thebibliography}{9}
\bibitem{dijkstra}
Wikipedia,
\textit{Dijkstra's algorithm.} \url{https://en.wikipedia.org/wiki/Dijkstra%27s_algorithm}.

\bibitem{mathlib4}
Lean4, Mathlib4,
\textit{SimpleGraph documentation.} \url{https://leanprover-community.github.io/mathlib4_docs/Mathlib/Combinatorics/SimpleGraph/Basic.html#SimpleGraph}.

\bibitem{clrs}
Thomas H. Cormen, Charles E. Leiserson, Ronald L. Rivest, and Clifford Stein,  
\textit{Introduction to Algorithms},  
3rd edition,  
MIT Press, Cambridge, MA, 2009.
\end{thebibliography}

\end{document}
